\documentclass{article}
\usepackage[utf8]{inputenc}
\usepackage{graphicx}
\usepackage{wrapfig}
\graphicspath{ {images/} }

\title{Math 150A Optimization Writing Project}
\author{David Feinzimer}
\date{June 25, 2017}

\begin{document}

\maketitle

\section{Description of Problem: The Least Expensive Path (Quickest Path)}

\begin{wrapfigure}{l}{0.25\textwidth}
    \centering
    \includegraphics[width=0.25\textwidth]{title}
\end{wrapfigure}

Our problem is that with a limited amount of time we must run across campus to reach class.
Our destination is 1,000 feet south and 200 feet east of our current location.
When traveling southbound we can move at a rate of 10 feet per second.
When traveling eastbound we can move at a rate of 3 feet per second.
We have about four different paths to choose from but only one path with allow for the shortest trip to reach class.

Path 1 would first take us 1,000 feet southbound at 10 feet per second. However, we'd still need to travel 200 feet eastbound ultimately making this a costly option.

Path 2 would first take us 200 feet eastbound at 3 feet per second. However, we'd still need to travel 1,000 southbound through the quad at the equally slow speed of 3 feet per second making this path even more costly than path 1.

Path 3 would take us directly towards the classroom (as the bird flys) on a southeastern path at a slow rate of 3 feet per second but there would be no need to travel directly southbound followed by directly eastbound or vice versa.

A fourth (4) and final path would take us on a mixture of a pure southbound 10 foot per second stint followed by a slower 3 foot per second southeaster stint.

In the end, path 4 would be the best route to take. The rest of this paper will explain why that is and how I arrived at this conclusion.

\section{Theory And Process To Be Followed}

Modern day GPS systems and way-finding software specialize in solving this exact type of problem. The following is a description of the almost certain method these systems follow in order to deliver the most direct and efficient directions to their client's everyday.

While these advanced systems need to deal with real time data like constantly changing traffic patterns, our instance of this problem is a bit simpler as we have defined rates of speed that will not dynamically and constantly change on us.

Step 1) \t I will recognize, process, and give a label to all of the information that has been provided.

Step 2) \t I will recognized and explain the different relevant equations and their derivatives that will allow us to land at a correct answer. In this case I want to minimize time spent traveling to my class in McCarthy Hall.

Step 3) \t I will show all of the work involved in solving this specific way-finding problem.

\section{Processing \& Labeling Given Data}

In solving this problem it will help us to analyze and label our data as if we were looking at it on an X/Y coordinate plane.
From the problem, we are told we need to travel 1,000 feet south. Due to the face that the Y-axis traditionally runs north/south I will officially recognize our starting point as 'S', an intersect on the Y-axis at (0,1000).
From the problem, we are also told we need to travel 200 feet east. Due to the face that the X-axis traditionally west/east I will recognize our ending point or destination as 'D', an intersect on the X-axis at (200,0). 
We are told a main walk-way runs north/south and allows travelers to move at a rate of 10 feet per second. Because the main walk-way runs north/south it lies along our Y-axis and because this is a rate of change on our Y-axis with respect to time we have $ \frac{dy}{dt} = 10\frac{ft}{sec} $ 
A quad also lies between us and class allowing travelers to only move at a rate of 3 feet per second. Because we will be traveling diagonally through the quad with reference to the Pythagorean Theorem which will be used later on this is a rate of change on our Z-axis with respect to time we have $ \frac{dz}{dt} = 3\frac{ft}{sec} $ 

\vspace{.5cm}
Given all of the above information we have the following variables:

\vspace{.25cm}
S @ (0,1000)

\vspace{.25cm}
D @ (200,0)

\vspace{.25cm}
$ \frac{dy}{dt} = 10\frac{ft}{sec} $

\vspace{.25cm}
$ \frac{dz}{dt} = 3\frac{ft}{sec} $ 

\section{Recognizing Relevant Equations}

\end{document}
